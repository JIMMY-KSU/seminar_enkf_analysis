\documentclass[a4paper,5p]{elsarticle}

\usepackage{graphicx}
\usepackage{amsmath}
\usepackage{amssymb}
\usepackage{amsthm}

\newcommand{\norm}[1]{\left\lVert#1\right\rVert}

\begin{document}

\begin{frontmatter}

\title{Analysis of the Ensemble Kalman Filter for Inverse Problems}
\author{Matthieu Bulté}
\ead{matthieu.bulte@tum.de}
\address{Technische Universität München}

\begin{abstract}
The EnKF (Ensemble Kalman Filter) is a widely adopted methodology
for state estimation in partially and noisily observed dynamical systems. While
its adaptation for estimating solutions of Bayesian Inverse Problems (BIPs)
has shown many positive empirical results, the theoretical understanding of
the method is still poor. In this paper, we summarize the rigorous analysis of
the EnKF proposed by C. Shillings and A. Stuart \cite{schillings2017analysis} and demonstrate their
theoretical results through numerical experiments.
\end{abstract}
\begin{keyword}
EnKF \sep Inverse Problems \sep Bayesian statistics \sep Optimisation
\end{keyword}
\end{frontmatter}

\section{Introduction} \label{sec:1} 
The ensemble Kalman filter (EnKF) has had a large impact in the
natural sciences over the past years. Using an ensemble of particles,
the method was orginally used to approximate the solution of data
assimilation problems \cite{iglesias2013ensemble}, but can also be extended to approximate
the solution of Bayesian inverse problems. While the method is well understood
in the large ensemble limit, the method is often used
by practitioners because of its robustness - even with a small number
of particles. It is thus relevant to study different properties of the
algorithm without considering the large ensemble limit.

This article is a review of the work of C. Shillings and A. Stuart \cite{schillings2017analysis}.
Section \ref{sec:2} presents the EnKF applied to inverse problems and will draw links
between the method and the optimization approach to inverse problems. A selection of theoretical
results will then be presented and proved in Section \ref{sec:3}, and empirically tested with numerical
experiments in Section \ref{sec:4}.

\section{A fresh look at the EnKF} \label{sec:2}
\subsection{Inverse Problems} \label{subsec:2:1}
In this section, we will consider the EnKF for solving inverse problems. Given a continuous
map $\mathcal{G} : \mathcal{X} \rightarrow \mathcal{Y}$ between two Hilbert spaces $\mathcal{X}$
and $\mathcal{Y}$, we would like to identify $u \in \mathcal{X}$ solving 
\begin{equation} \label{eq:2:1:1}
    y = \mathcal{G}(u) + \eta,
\end{equation}
where $\eta$ is the \textit{observational noise}. For simplicity, we will only consider the case where $\mathcal{Y}$
is a finite dimensional space, but the result presented can mostly be extended to support general Hilbert spaces.
A central element in solving inverse problems is the \textit{least squares functional} $\Phi : \mathcal{X} \times \mathcal{Y} \rightarrow \mathbb{R}$, measuring
the model-data misfit
given by
\begin{equation} \label{eq:2:1:2}
    \Phi(u; y) = \frac12\norm{\Gamma^{-1/2}\left(y - \mathcal{G}(u)\right)}_\mathcal{Y}^2.
\end{equation}
In this expression, $\Gamma$ is a positive semi-definite operator describing the covariance
structure of the observational noise $\eta$. In the general case, this problem is \textit{ill-posed}
and minimization of the data-model misfit through minimization of the least squares operator is not
possible. However, solvability of this system can be improved by use of \textit{regularization} methods.

As thouroughly described by A. Stuart \cite{stuart2010inverse} one successful approach to regularization of inverse problems
is to consider the problem from a Bayesian point of view. There, the problem in (\ref{eq:2:1:1}) is replaced
by considering the pair $(u, y)$ as a joint random variable over $\mathcal{X} \times \mathcal{Y}$. Assuming
that we can encode current knowledge about $u$ in a \textit{prior} distribution $\mu_0$ with $u \sim \mu_0$
independent of the observational noise $\eta \sim \text{N}(0, \Gamma)$, the solution of the
\textit{Bayesian inverse problem} is given by $u|y \sim \mu$ with
\begin{equation} \label{eq:2:1:3}
    \mu(\text{d}u) \propto \exp\left(-\Phi(u; y)\right)\mu_0(\text{d}u).
\end{equation}

We now investigate how the EnKF method can be used to estimate the solution given in (\ref{eq:2:1:3}) of the 
inverse problem.

\subsection{The EnKF for inverse problems} \label{subsec:2:2}
Since the EnKF algorithm is designed to solve data assimilation problem, an artificial dynamic must be
defined in order to apply the method to inverse problems. To that end, we consider a discrete dynamic
over the space $\mathcal{X} \times \mathcal{Y}$ defined by the operator
\begin{equation} \label{eq:2:2:1}
    \Xi(u, p) = (u, \mathcal{G}(u)).
\end{equation}
We then complete this artificial process by defining the observational process, such that for each
time step $n \in \mathbb{N}$ with associated state $(u_n, p_n)$, the observation $y_n$ is given by
\begin{equation} \label{eq:2:2:2}
    y_n = p_n + \xi_n,
\end{equation}
where $\{\xi_j\}_{j\in \mathbb{N}}$ is an i.i.d. sequence distributed according to $\text{N}(0, \Sigma)$,
in which $\Sigma$ is a positive semidefinite matrix.

The EnKF can then be used to estimate each step $n \in \mathbb{N}$ of the dynamical system using an
unweighted set of $J$ particles $\{u_n^{(j)}\}_{j = 1}^J$, constructing the empirical measure
\begin{equation} \label{eq:2:2:3}
    \mu_n \approx \hat{\mu}_n := \frac1J\sum_{j=1}^J \delta_{u_n^{(j)}}.
\end{equation}

Replacing the dynamic (\ref{eq:2:2:2}) in the standard update formulate of the EnKF gives the
following update rule for the ensemble of particles
\begin{equation} \label{eq:2:2:4}
    u_{n+1}^{(j)} = u_{n}^{(j)} + C^{\text{up}}(u_n)[C^{\text{pp}}(u_n) + h^{-1}\Gamma]^{-1}(y_{n+1}^{(j)} - G(u_n^{(j)}))
\end{equation}
with 
\begin{equation}
    \begin{aligned}
        C^{\text{up}} &= \hat{\text{cov}}(u, \mathcal{G}(u)),\\
        C^{\text{pp}} &= \hat{\text{cov}}(\mathcal{G}(u), \mathcal{G}(u)),
    \end{aligned}
\end{equation}
where $\hat{\text{cov}}$ is the empirical covariance computed from the approximate measure $\hat{\mu}_n$.
We can then consider cases where $\Sigma = \Gamma$, or where $\Sigma = 0$, i.e. where no artificial noise
is added to the data. In this article, we will only consider the second case.

This application of the EnKF for solving inverse problems can be shown to properly approximate the
distribution $\mu$ of $u|y$ in the large ensemble limit when the model $\mathcal{G}$ is linear and
the noise $\eta$ is Gaussian, see \cite{goldstein2007bayes, law2016deterministic}. However, since the
EnKF relies on a linear approximation, the error term resulting from the model estimation is independent
of the number of particles, see \cite{ernst2015analysis}.

\section{Study of the continous time limit} \label{sec:3}
\subsection{EnKF as a gradient flow} \label{subsec:3:1}
While we have mentionned some properties of the large ensemble limit, it is also interesting to consider
what happens for $h \rightarrow 0$, that is, we want to consider the continuous time limit
of the algorithm. With that in mind, we can take a carefull look at the update rule (\ref{eq:2:2:4}) and
recognize that the equation corresponds to an Euler-Maruyama discretization of the following coupled
system of SDEs
\begin{equation} \label{eq:3:1:1}
    \frac{\text{d}u^{(j)}}{\text{d}t} = C^{\text{up}}(u)\Gamma^{-1}\left(y - \mathcal{G}(u^{(j)}) + \sqrt{\Sigma}\frac{\text{d}W^{(j)}}{\text{d}t}\right),
\end{equation}
where $\{W^{(j)}\}_{j=1}^J$ are independent Brownian motions. 

Furthermore, assuming $\mathcal{G}(u) = Au$ is a linear map and $\Sigma = 0$, we obtain a deterministic
system of coupled ODEs which, given the definition of the empirical covariance is
\begin{equation} \label{eq:3:1:2}
\begin{aligned}
    \frac{\text{d}u^{(j)}}{\text{d}t} &= \hat{\text{cov}}(u, Au)\Gamma^{-1}(y - Au^{(j)})\\
&= \frac1J\sum_{k=1}^J \langle A(u^{(k)} - \overline{u}), y - Au^{(j)}\rangle_\Gamma(u^{(k)} - \overline{u})\\
&= A^T\Gamma^{-1}(y - Au^{(j)}) \hat{\text{cov}}(u, u)\\
&= -\hat{\text{cov}}(u, u)\text{D}_u\Phi(u^{(j)}; y).
\end{aligned}
\end{equation}

This shows that in its continuous time limit, the EnKF algorithm performs a set of gradient descents
for $\Phi(\cdot; y)$, coupled through a preconditioning of the ensemble by its empirical covariance. This
result draws a link between the optimization and Bayesian approaches to inverse problems.

\subsection{Properties of the gradient flow} \label{subsec:3:2}

We are now interested in studying properties of the coupled gradient flows presented
in the previous section.

\section{Numerical results} \label{sec:4}

\section{Conclusion} \label{sec:5}

\bibliographystyle{model1-num-names}
\bibliography{../literature/literature.bib}

\end{document}
